\documentclass{article}

\title{Physical based rendering cheat sheet} 
\author{Maksim Jaroslavcevas}
\date{\today}

\begin{document}

\maketitle

\section{Rendering equation}
\[
L_{o}(\textbf{x}, \omega_{o}, \lambda, t) = L_{e}(\textbf{x}, \omega_{o}, \lambda, t) + L_{r}(\textbf{x}, \omega_{o}, \lambda, t)
\]
where

\begin{itemize}
    \item $L_{o}$ is the total spectral radiance of wavelength $\lambda$ directed outward along direction $\omega_{o}$ at time t, from a particular position $x$
    \item $\textbf{x}$ is the world position
    \item $\omega_{o}$ is direction of the outgoing light
    \item $\lambda$ is a light wavelength
    \item $t$ is time
    \item $L_{e}(\textbf{x}, \omega_{o}, \lambda, t)$ is emitted spectral radiance
    \item $L_{r}(\textbf{x}, \omega_{o}, \lambda, t)$ is reflected spectral radiance
\end{itemize}

Since we will use this equation in computer graphics scope we can, eliminate irrelevant 
variables, such as $\lambda$ and $t$. Result equation:

\begin{equation}
    L_{o}(\textbf{x}, \omega_{o}) = L_{e}(x, \omega_{o}) + L_{r}(\textbf{x}, \omega_{o})
\end{equation}

As a result we get very neat function. This functions gives a color of a point at position \textbf{x} 
from view vector perspective, as we see final color is a sum of a emitted light and reflected light.

\textit{First and second functions work with only single light channel, 
but actually if we calculate each of RGB channel separately and compose 
into a single three-dimensional vector, result will be the same as just 
replacing evrething with three-dimensional vector. Next we will consider 
light as a three-dimensional vector, where each dimension corresponds to
red, green blue channels respectively.}

Now lets look at reflection function

\subsection{Emission function}
Todo

\subsection{Reflection function}
\[
L_{r}(x, \omega_{o}, \lambda, t) = \int _{\Omega } f_r(x, \omega_{i}, \omega_{o}, \lambda, t) L_{i}(x, \omega_{i}, \lambda, t) (\omega_{i} * N) dw_{i}
\]
where
\begin{itemize}
    \item $\textbf{N}$ is a surface normal
    \item $\omega_{i}$ light vector
    \item $\cdots$ all other variables are repeated as in General form equation
    \item $L_{i}(x, \omega_{i}, \lambda, t)$ incoming light 
\end{itemize}

Lets simplify this function, eliminating $\lambda, t$ variables

\[
L_{r}(x, \omega_{o}) = \int _{\Omega } f_r(x, \omega_{i}, \omega_{o}) L_{i}(x, \omega_{i}) (\omega_{i} * N) dw_{i}
\]


As wee see reflected light is just an $\int$ of all incoming light from an $\Omega$ 
sphere. Since in computer graphics we usually have a finite number of lights, we can replace $\int$ with a $\Sigma$.

\begin{equation}
    L_{r}(x, \omega_{o}) = \sum_{n = 1}^{N} f_r(x, \omega_{n}, \omega_{o}) L_{i}(x, \omega_{n}) (\omega_{n} * N)
\end{equation}

Now we get reflection is a sum of products between incoming lights, $f_r$ functions and $(\omega_{n} * N)$. 
At this point our final equation should look something like this

\[
L_{o}(\textbf{x}, \omega_{o}) = L_{e}(x, \omega_{o}) + \sum_{n = 1}^{G} f_r(x, \omega_{n}, \omega_{o}) L_{i}(x, \omega_{n}) (\omega_{n} * N)
\]
\begin{itemize}
    \item $\textbf{G}$ total number of lights
\end{itemize}

Now we need to clarify $f_r$ function, but before lets replace some variables, 
since we will use them quite a lot further, and not all names are quite intuitive.

\begin{equation}
    L_{o}(x, V) = L_{e}(x, V) + \sum_{n = 1}^{G} f_r(x, L_{n}, V) L_{i}(x, L_{n}) (L_{n} * N)
\end{equation}

where
\begin{itemize}
    \item $\textbf{x}$ is a world position of a point
    \item $\textbf{V}$ view vector
    \item $L_{n}$ - $n^{th}$ light vector
    \item $N$ surface normal
    \item $\textbf{G}$ total number of lights
    \item $L_{i}$ - incoming light function
\end{itemize}

Note that all vectors are normalized.

\subsection{Bidirectional reflectance distribution function (BRDF)}

BRDF stands by bidirectional reflectance distribution function, this function
defines how light from a source is reflected off an opaque surface. Usually BRDF
takes an incoming light direction and outgoing direction, and returns the ration of
reflected radiance exiting along to the irradiance incident of the surface from 
view direction.

Lets take a look at first Fred Nicodemus definition of the BRDF:

\begin{equation}
    f_r(\omega_{i}, \omega_{r}) = \frac{dL_r(\omega_{r})}{dE_i(\omega_{i})} = \frac{dL_r(\omega_{r})}{dL_i(\omega_{i})cos(\theta_{i})d\omega_{i}}
\end{equation}

where 
\begin{itemize}
    \item $\textbf{L}$ is radiance
    \item $\textbf{E}$ is irradiance
    \item $\theta_{i}$ is the angle between $\omega_{i}$ and the surface normal
    \item $i$ index indicates incident light
    \item $r$ index indicates reflected light
\end{itemize}

While the BRDF itself is a general concept that can model any type 
of reflection, but it is not very suitable for our needs, 
since in computer graphics we usually interested in
diffuse and specular reflections. There are many other BRDF mased light
models, such as

\begin{itemize}
    \item $\textbf{Lambertian model}$ used for diffuse reflection
    \item $\textbf{Cook-Torrance model (Torrance–Sparrow)}$ specular-microfacet model
    \item $\textbf{Phong}$ This model adds a specular component to the Lambertian model. Computationally inexpensive but less physically accurate
    \item $\textbf{Oren–Nayar}$ similar to Lambertians model, more physically accurate, but more computationally expensive
    \item ... and many others
\end{itemize}

Since different lighting models are used for different reflections,
we can use combination of different models, for better result. For example most common use is a
$\textbf{Lambertian model}$ for diffuse reflections and $\textbf{Cook-Torrance model}$ for specular reflections. 
But we need to consider energy preservation, outgoing energy should not exceed incoming energy.

But in what ratio we need to sum our diffuse and specular relfections? For this purpose
we can use $\textbf{Fresnel}$ effect. In computer graphics we usually use 
$\textbf{Schlick's approximation}$ for approximating the constribution of the $\textbf{Fresnel}$ effect in the
specular reflection.

Original $\textbf{Schlick's approximation}$ function

\[
    f_{Schlick}(\theta) = f_{0} + (1 - f_{0})(1 - \cos{\theta})^5
\]

where
\begin{itemize}
    \item $f_{0}$ is the reflection coefficient for light incoming parallel to the normal
    \item $\theta$ is the angle between the direction from which the incident light is comming and the normal
\end{itemize}

Since in all our previous calculations we had no $\theta$ angle, lets replace it with vector dot product.

\[
    f_{Schlick} = f_{0} + (1 - f_{0})(1 - (V \cdot H))^5
\]

where
\begin{itemize}
    \item $V$ view vector
    \item $H$ half-way vector
\end{itemize}

Note that $V$ and $H$ vectors are normalized, and $f_{0}$ is a constant specific for a specific material.

Since $\textbf{Schlick's approximation}$ gives a ratio of specular reflection component lets denote it as

\[
    k_s = f_{Schlick}
\]
\[
    k_d = 1 - k_s
\]

Note that $k_s + k_d = 1$ ($k_s$ - specular ration, $k_d$ - diffuse ratio).

At this point our $BRDF$ (Combination of Lambertian model and Cook-Torrance model for diffuse and specular reflections respectively) function, 
should look something like this

\[
    f_{brdf} = k_d f_{lambert} + k_s f_{cook-torrance}
\]

\subsection{Lambertian model}

\[
    f_{lambert} = \frac{Color}{\pi}(L_{n} * N)
\]

From where does $\pi$ come from?

Since Lambertian diffuse model describes perfect diffuse reflections
(incoming light is equally reflected in all hemisphere directions).
Therefore incoming light is equal to light reflected in all hemisphere directions.
Note that there we have $\textbf{'all hemisphere directions'}$

\[
    L_i = \int _{\Omega } L_o
\]

where
\begin{itemize}
    \item $L_i$ incoming light
    \item $L_o$ outcomming light
    \item $\Omega$ our hemisphere
\end{itemize}

We will not go deeper into hemisphere integration, but note that
there we also need to normalize our hemisphere integral. As a result we get $\frac{1}{/pi}$.

Since lambertian model describes perfect diffuse relfections,
therefore viewing angle and surface normal are not
really important, therefore they can be simplified 

\[
    f_{lambert} = \frac{Color}{\pi}
\]

As a result our final lambertian diffuse equation have been simplified to a constant value.

\subsection{Cook-Torrance model}
Todo

\section{Final result}
Todo

\section{References}
Todo

\end{document}
